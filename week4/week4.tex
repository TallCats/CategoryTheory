\documentclass[11pt]{article}
\usepackage[utf8]{inputenc}
\usepackage[T1]{fontenc}
\usepackage{graphicx}
\usepackage{amsmath}
\usepackage{amssymb}
\usepackage{hyperref}
\usepackage{ 
  %   newtxtext
  % , newtxmath
  , minted
  , lmodern
  , listings
  , xparse
  , amsthm
  , inconsolata
  , marvosym
  }
\usepackage[all,2cell]{xy}\UseAllTwocells

\def\opp{\text{op}}
\def\Set{\mathsf{Set}}
\def\BF#1{\mathsf{#1}}

\def\fold{{\sf fold}}
\usepackage{enumitem}

\setlist[itemize]{itemsep=0pt, topsep=4pt}
\DeclareMathOperator{\colim}{colim}

\setminted{fontfamily=tt}
\usemintedstyle{tango}

% \usepackage[left=5.5cm,right=5.5cm]{geometry}

\newcommand{\smat}[1]{\left[\begin{smallmatrix}#1\end{smallmatrix}\right]}

\newcommand{\mil}[1]{\mintinline{haskell}{#1}}
\usepackage{CJKutf8}

\newcommand{\jap}[1]{\text{\begin{CJK}{UTF8}{min}#1\end{CJK}}}

\newtheorem{theorem}{Theorem}
\newtheorem{definition}{Definition}
\newtheorem{example}{Example}
\newtheorem{notation}{Notation}
\newtheorem{remark}{Remark}
\newtheorem{lemma}{Lemma}
\ExplSyntaxOn
\NewDocumentCommand{\makeabbrev}{mmm}
 {
  \yoruk_makeabbrev:nnn { #1 } { #2 } { #3 }
 }

\cs_new_protected:Npn \yoruk_makeabbrev:nnn #1 #2 #3
 {
  \clist_map_inline:nn { #3 }
   {
    \cs_new_protected:cpn { #2 } { #1 { ##1 } }
   }
 }
\ExplSyntaxOff

\makeabbrev{\textbf}{bf#1}{
  a,b,c,d,e,g,h,i,j,k,l,m,n,o,p,q,r,t,u,v,w,x,y,z,%
  A,B,C,D,E,G,H,I,J,K,L,M,N,O,P,Q,R,T,U,V,W,X,Y,Z }

\makeabbrev{\boldsymbol}{bs#1}{%
    a,b,c,d,e,f,g,h,i,j,k,l,m,n,o,p,q,r,s,t,u,v,w,x,y,z,%
    A,B,C,D,E,F,G,H,I,J,K,L,M,N,O,P,Q,R,S,T,U,V,W,X,Y,Z }

\makeabbrev{\mathsf}{sf#1}{
  a,b,c,d,e,f,g,h,i,j,k,l,m,n,o,p,q,r,s,t,u,v,w,x,y,z,%
  A,B,C,D,E,F,G,H,I,J,K,L,M,N,O,P,Q,R,S,T,U,V,W,X,Y,Z }
\makeabbrev{\mathfrak}{fk#1}{
  a,b,c,d,e,f,g,h,j,k,i,l,m,n,o,p,q,r,s,t,u,v,w,x,y,z,%
  A,B,C,D,E,F,G,H,I,J,K,L,M,N,O,P,Q,R,S,T,U,V,W,X,Y,Z }
\makeabbrev{\mathcal}{cl#1}{
  A,B,C,D,E,F,G,H,I,J,K,L,M,N,O,P,Q,R,S,T,U,V,W,X,Y,Z }
\makeabbrev{\mathbb}{bb#1}{
  A,B,C,D,E,F,G,H,I,J,K,L,M,N,O,P,Q,R,S,T,U,V,W,X,Y,Z }

\hypersetup{ linktoc    = all
           , colorlinks = true
           , urlcolor   = green!50!black
           , citecolor  = green!50!black
           , linkcolor  = green!50!black
           }
\author{ Fosco Loregian \\
  \href{mailto:fouche@yoneda.ninja}
       {\tt fouche@yoneda.ninja}}
\date{\today}
\title{  Category theory course 
			\\ Lecture 4
			\\ ITI9200, Spring 2020
			}
\setcounter{tocdepth}{2}
\begin{document}

\maketitle
\tableofcontents
\section{Functors}
\subsection{Morphisms of categories}
\begin{definition}[Functor]
	Let $\clC$ and $\clD$ be two categories; we define a \emph{functor} $F :\clC\to \clD$ as a pair $(F_0, F_1)$ consisting of the following data:
	\begin{itemize}
		\item  $F_0$ is a function $ \clC_o\to  \clD_o$ sending an object $C\in \clC_o$ to an object $FC \in \clD_o$;
		\item  $F_1$ is a family of functions $F_{AB} : \clC(A,B)\to \clD(FA,FB)$, one for each pair of objects $A,B\in\clC_o$, sending each arrow $f:A\to B$ into an arrow $Ff:FA\to FB$, and such that:
		\begin{itemize}
			\item $F_{AA} (1_A)=1_{F A}$;
			\item $F_{AC} (g\circ_\clC f)=F_{BC} (g)\circ_\clD F_{AB} (f)$.
		\end{itemize}
	\end{itemize}
\end{definition}
\begin{minted}{haskell}
-- | All instances of the `Functor` type-class must satisfy two laws.
-- These laws are not checked by the compiler.
--
-- * The law of identity
--   `forall x. (id <$> x) ~ x`
--
-- * The law of composition
--   `forall f g x.(f . g <$> x) ~ (f <$> (g <$> x))`
class Functor k where
-- Pronounced, eff-map.
(<$>) :: (a -> b) -> k a -> k b
\end{minted}
If $F : \clC \to \clD$ is a functor, it corresponds to a data constructor \mil{f :: * -> *} (this is  the $F_0$ part, a \emph{function on objects} from the collection of object $\clC_o$ of $\clC$ to the collection of objects $\clD_o$\footnote{Although in Haskell, $\clC=\clD$ is always the same nameless category \mil{*}.}) and a \emph{function on morphisms}
\begin{minted}{haskell}
<$> :: (a -> b) -> f a -> f b
\end{minted}
If you recall how the associativity of \mil{->} works, it is evident that this is a function from the type of maps \mil{a -> b} to the type of maps \mil{f a -> f b}; or rather, this is a function that given a function \mil{u :: a -> b}, and an ``element'' of type \mil{f a}, yields an element \mil{x :: f b}.
\begin{remark}
	There is a category whose objects are (small) categories, and whose morphisms $\clC\to \clD$ are functors. The \emph{identity} functor $1_{\clC} : \clC \to \clC$ of a category $\clC$, and the composition $G\circ F : \clA \xrightarrow{F} \clB \xrightarrow{G} \clC$ of two functors are defined in the obvious way; all category axioms follow.\footnote{The reader might now wonder if there is a category of Haskell types; unfortunately there is no such thing, but better languages have a better behaviourin this respect (for example \mil{Agda} or \mil{Idris}).}
\end{remark}
\subsection{Examples of functors}
As you can imagine, both Haskell and mathematics are crawling with examples of functors;
\begin{example}[Examples of functors]\leavevmode
	\begin{enumerate}
		\item  Let's rule out a few edge examples: for every category $\clC$, there is a unique functor $F : \varnothing \to \clC$, where $\varnothing$ is the empty category with no objects an no morphisms; for every category $\clC$, there is an obvious definition of the \emph{identity functor} $1_\clC : \clC \to \clC$, whose correspondences on objects and on arrows $1_{\clC, AB} : \clC(A,B) \to \clC(A,B)$ are all identity functions.
		\item  As we have hinted in the preceding lesson, all correspondences that \emph{forget} a structure, yielding for example the underlying set $U\bbM$ of a thingum $\bbM$, are functors ${\sf Thng} \to {\sf Set}$ from their structured categories of definition to the category of sets and functions.
		\item  All correspondences that \emph{regard} a structure as an example of another are -more or less tautological- examples of functors: a monoid can be seen as a one-object category, and this yields a functor $\bfB : {\sf Mon} \to {\sf Cat}$ because a functor $F : \bfB M \to \bfB N$ between monoids is but a monoid homomorphism $f : M \to N$; a preset can be seen as a category with at most one arrow between any two objects, and this defines a functor $^\chi(\_) : {\sf Pres} \to {\sf Cat}$, because a functor $F : {}^\chi P \to {}^\chi Q$ between presets is but a monotone function $f : P \to Q$; a set can be seen as a discrete topological space $A^\delta$ or as a category having only identity arrows, and these are functors ${\sf Set}\to {\sf Top}, {\sf Cat}$. (Continue the list at your will.)
		\item  Let $M$ be a monoid, and $\boldsymbol{\Delta}$ be the category having objects nonempty, finite, totally ordered sets and monotone functions as morphisms. The correspondence that sends $[n]\in\boldsymbol{\Delta}$ to the set $M^{[n]}$ of ordered $n$-tuples $[a_1|\dots|a_n]$ of elements of $M$ and a monotone function $f : [m]\to [n]$ to the function $M^f : M^{[n]}\to M^{[m]}$ defined by
		\[ M^f[a_1|\dots|a_n] = [a_{f1}|\dots|a_{fn}] \]
		is a functor $\boldsymbol{\Delta}^\opp \to \Set$. This functor is called the \emph{classifying space} of the monoid.
		\item  Let $\Set_*$ be the category of \emph{pointed sets}: objects are pairs $(A,a)$ where $a\in A$ is a distinguished element, and a morphism $f : (A,a) \to (B,b)$ is a function $f :A\to B$ such that $fa=b$. It is easy to prove that this category is isomorphic to the coslice  $*/\Set$ of functions $a : * \to A$.
		\item  Let $\partial\Set$ be the category so defined: objects are sets, and a morphism $(f,S) : A \to B$ is  a pair $(S \subseteq A, f : S \to B)$. This is called a \emph{partial function} from $A$ to $B$. It is evident how to define the composition of morphisms, and the identity arrow of $A \in\partial\Set$. There is a functor $(\_)_\bullet : \partial\Set \to \Set_*$ defined as $A\mapsto (A\sqcup \{\infty\}, \infty)$, where $\infty$ is an element that does not belong to $A$, and a function $(f,S)$ goes to $(f,S)_\bullet : A\sqcup \{\infty_A\} \to B\sqcup\{\infty_B\}$, defined sending $S^c\cup \{\infty_A\}$ to $\{\infty_B\}$.
		\item  Every directed graph $\underline G$, with set of vertices $V$ and set of edges $E$, defines a quotient set of $V$ by the equivalence relation $\approx$ generated by the subset of those $(A, B)$ for which there is an arrow $A\to B$; the symmetric and transitive closure of this relation yields a quotient $V/R$ that is usually denoted as the set of \emph{connected components} $\pi_0(\underline G)$. This is a functor $\BF{Gph} \to \Set$, and if we now regard a category $\clC$ as a mere directed graph we obtain a well-defined set $\pi_0(\clC)$ of connected components of a category.
	\end{enumerate}
\end{example}

\subsection{Diagrams or: functors as pictures}
Most of category theory is based on the intuition that a diagram of a certain shape in a category $\clC$ is a functor $F : \clJ \to \clC$; the definition of a diagram itself is engineered to blur the distinction between a functor as a morphism of categories, and a functor as a correspondence that ``pictures'' a small category $\clJ$ inside a big category $\clC$. After having recalled the definition of diagram, and the definition of commutative diagram, we collect a series of examples to convince the reader that this identification of ``diagrams as pictures'' is fruitful and suggestive.

The recommended soundtrack while reading this section is, of course, Emerson, Lake \& Palmer's \emph{Pictures at an Exhibition}.

\subsubsection{Diagrams, formally}
Arrangements of objects and arrows in a category are called \emph{diagrams}; to some extent, category theory is the art of making diagrams \emph{commute}, i.e. the art of proving that two paths $X\to A_1\to A_2 \to\cdots\to A_n \to Y$ and $X \to B_1\to\cdots\to B_m \to Y$ result in the same arrow when they are fully composed.

\begin{remark}[Diagrams and their commutation]\leavevmode
	Depicting morphisms as arrows allows to draw regions of a given category $\clC$ as parts of a (possibly non planar) graph; we call a \emph{diagram} such a region in $\clC$, the graph whose vertices are objects of $\clC$ and whose edges are morphisms of suitable domains and codomains. For example, we can consider the diagram
	\[
		\vcenter{\xymatrix{
				&&X_2\ar[drr]^v&& \\
				X_1\ar[dr]_g\ar[urr]^u\ar[rrrr]|(.375)\hole_(.7)h&&&&X_3\\
				& X_0 \ar[uur]_(.3)q\ar[rr]_p&& X_4 \ar[ur]_k&
			}}
	\]
	The presence of a composition rule in $\clC$ entails that we can meaningfully compose \emph{paths} $[u_0,\dots, u_n]$ of morphisms of $\clC$. In particular, we can consider diagrams having distinct paths between a fixed source and a fixed `sink' (say, in the diagram above, we can consider two different paths $\fkP = [k,p,g]$ and $\fkQ = [v,u]$); both paths go from $X_1$ to $X_3$, and we can ask the two compositions $\circ[k,p,g] = k\circ p\circ g$ and $v\circ u$ to be the same arrow $X_1\to X_3$; we say that a diagram \emph{commutes at $\fkP,\fkQ$} if this is the case; we say that a diagram \emph{commutes} (without mention of $\fkP, \fkQ$) if it commutes for every choice of paths for which this is meaningful.
\end{remark}
Searching a formalisation of this intuitive pictorial idea leads to the following:
\index{Diagram}
\begin{definition}
	A \emph{diagram} is a map of directed graphs (`digraphs') $D : J \to \clC$ where $J$ is a digraph and $\clC$ is the digraph underlying a category.\footnote{Every small category has an underlying graph, obtained keeping objects and arrows and forgetting all compositions; there is of course a category of graphs, and regarding a category as a graph is another example of forgetful functor. Of course, making this precise means that the collection of categories and functors form a category on its own right.} Such a diagram $D$ \emph{commutes} if for every pair of parallel edges $f,g : i\rightrightarrows j$ in $J$ one has $Df = Dg$.
\end{definition}
\subsubsection{Diagrams, everywhere}
\begin{itemize}
\item An object is a diagram: the terminal object in the category of categories is the category 
\[[0] = \Big\{ 
\vcenter{\xymatrix{
	\bullet \ar@(ur,dr)^1
}}\Big\}\]
A functor $C : \textbf{1} \to \clC$ consists of an object of $\clC$.
\item An arrow is a diagram: let 
\[[1] = \Big\{ \vcenter{\xymatrix{0 \ar[r] & 1}} \Big\}\]
be the category with two objects $0,1$ and a single nonidentity morphism $0\to 1$. A functor $f : \textbf{2} \to \clC$ consists of two objects $X_i = f(i)$ for $i=0,1$, and a morphism $X_0\to X_1$.
\item A chain is a diagram: let $n>2$ be a positive integer. Let 
\[[n] = \Big\{ \vcenter{\xymatrix{0 \ar[r] & 1 \ar[r] & \cdots \ar[r] & n}} \Big\}\] be the category with exactly $n+1$ objects and exactly one nonidentity arrow $j-1\to j$ for $j=1,\dots,n$. A functor $c : [n] \to \clC$ consists of a tuple of objects $X_0 \to X_1 \to \cdots \to X_n$, and morphisms $X_{j-1}\to X_j$ for $j=1,\dots,n$. 
\item A span is a diagram: let 
\[
\clS = \left\{ \vcenter{\xymatrix@R=4mm@C=4mm{
	&0\ar[dr]\ar[dl]& \\
	1 && 2
}}	\right\}
\] be the category with three objects and arrows as depicted. A functor $F : \clS \to \clC$ is a diagram of the same shape, labeled in the same way\dots

So, you get the idea. Define a \emph{cospan} in $\clC$ to be a functor $F$ with domain the opposite category $\clS^\opp$.
\item A square is a diagram:
\item A cube is a diagram:
\item A diagram is a diagram is a diagram,\dots, is a diagram
\end{itemize}
\subsection{General definition of limits and colimits}
\begin{definition}[Cone completions of $\clJ$]
	Let $\clJ$ be a small category; we denote $\clJ^\rhd$ the category obtained adding to $\clJ$ a single terminal object $\infty$; more in detail, $\clJ^\rhd$ has objects $\clJ_o \cup\{\infty\}$, where $\infty\notin \clJ$, and it is defined by
	\begin{align*}
		{\clJ^\rhd}(J,J')     & = \clJ(J,J') \\
		{\clJ^\rhd}(J,\infty) & = \{*\}
	\end{align*}
	and it is empty otherwise. This category is called the \emph{right cone} of $\clJ$.

	Dually, we define a category $\clJ^\lhd$, the \emph{left cone} of $\clJ$, as the category obtained adding to $J$ a single \emph{initial} object $-\infty$; this means that ${\clJ^\lhd}(J,J')=\clJ(J,J')$, ${\clJ^\lhd}(-\infty, J)=\{*\}$, and it is empty otherwise.
\end{definition}
\begin{example}
	Let $\omega$ be the ordinal number obtained from the union of all finite ordinals; then, when regarded as a category, $\omega^\lhd$ is the category $\{-\infty \to 0 \to 1\to \dots\}$ (and thus is isomorphic to $\omega$ in an evident way), whereas $\omega^\rhd$ is $\omega+1$, in the sense of ordinal sum.

	For those willing to embark in an extra exercise: what is $G^\rhd$ if $G$ is a monoid regarded as a category?
\end{example}
\begin{remark}
	The correspondences $\clJ\mapsto \clJ^\rhd$ and $\clJ\mapsto \clJ^\lhd$ are functorial. As an easy exercise, define them on morphisms and prove their functoriality. Prove also that there is a natural embedding $i_\rhd$ of $\clJ$ into $\clJ^\rhd$ and one $i_\lhd$ of $\clJ$ into $\clJ^\lhd$ (that we invariably denote $i$ in the following discussion).
\end{remark}
\begin{definition}[Cone of a diagram]
	Let $\clJ$ be a small category, $\clC$ a category, and $D : \clJ \to \clC$ a functor; all along this section, an idiosyncratic way to refer to $D$ will be as a \emph{diagram of shape $\clJ$}. We call a \emph{cone for $D$} any extension of the diagram $D$ to the left cone category of $\clJ$ defined in , so that the diagram
	\[
		\vcenter{\xymatrix{
				\clJ \ar[r]^D \ar[d]_{i_\lhd} & \clC \\
				\clJ^\lhd\ar@{.>}[ur]_{\bar D}
			}}
	\]
	commutes.
\end{definition}
Every such extension is thus forced to coincide with $D$ on all objects in $\clJ\subseteq \clJ^\lhd$; the value of $\bar D$ on $-\infty$ is called the base of the cone; dually, the value of an extension of $D$ to $\clJ^\rhd$ coincides with $D$ on $\clJ\subseteq \clJ^\rhd$, and $\bar D(\infty)$ is called the \emph{tip} of the cone.

There is of course a similar definition of a \emph{cocone} for $D$: it is an extension of the diagram $D$ to the right cone category of $\clJ$ so that the diagram
\[
	\vcenter{\xymatrix{
			\clJ \ar[r]^D \ar[d]_{i_\rhd} & \clC \\
			\clJ^\rhd\ar@{.>}[ur]_{\bar D}
		}}
\]
commutes. Exercise  sheds a light on  and it substantiates the quote from \cite{acc} therein: cones for $D$ are exactly cocones for the opposite functor $D^{op}$.
\begin{remark}\leavevmode
	\begin{itemize}
		\item The class of cones for $D$ forms a category $Cn(D)$, whose morphisms are the natural transformations $\alpha : D'\Rightarrow D'' : \clJ \to \clC$ such that the right whiskering of $\alpha$ with $i : \clJ \to \clJ^\lhd$ coincides with the identity natural transformation of $D$; this means that a morphism $\alpha$ of this sort is a natural transformation such that
		      \[
			      \vcenter{\xymatrix{
					      &\clJ\ar[dl]_i\ar[dr]^D & \\
					      \clJ^\lhd \rrtwocell^{D'}_{D''}{\alpha}&& \clC
				      }}	\quad = 1_D
		      \]
		      as a 2-cell
		      $D\Rightarrow D$.	\item Dually, the class of cocones for $D$ forms a category $Ccn(D)$, whose morphisms are the natural transformations $\alpha : D'\Rightarrow D'' : \clJ \to \clC$ such that the right whiskering of $\alpha$ with $i : \clJ \to \clJ^\rhd$ coincides with the identity natural transformation of $D$; this means that a morphism $\alpha$ of this sort is a natural transformation such that
		      \[
			      \vcenter{\xymatrix{
					      &\clJ\ar[dl]_i\ar[dr]^D & \\
					      \clJ^\rhd \rrtwocell^{D'}_{D''}{\alpha}&& \clC
				      }}	\quad = 1_D
		      \]
		      as a 2-cell $D\Rightarrow D$.
	\end{itemize}
\end{remark}
\begin{definition}[Colimit, limit]
	The \emph{limit} of a diagram $D : \clJ\to \clC$ is the terminal object denoted `$\lim_\clJ D$' in the category of cones for $D$; dually, the \emph{colimit} of $D$ is the initial object denoted `$\colim_\clJ D$' in the category of cocones for $D$.
\end{definition}
It is a good idea to unwind definitions  and  in order to obtain the more classically explained notion of limit and colimit: the key for this unwinding operation is that a co/cone for $D : \clJ \to \clC$ amounts to a natural family of maps from a constant object (the base) or to a constant object (the tip).
\begin{itemize}
	\item A \emph{cone} for a diagram $D : \clJ \to \clC$ is a natural transformation from a constant functor $\Delta_c : \clJ \to \clC$ to $D(\_)$;
	\item there is a category of cones for $D(\_)$, where morphisms between a cone $c\to D(\_)$ and a cone $C'\to D(\_)$ are arrows $k : C\to C'$ such that the diagram
	      \[
		      \vcenter{\xymatrix{
				      DI \ar[dd]_{D\phi}&& \\
				      &C\ar@{.>}[r]^k \ar[ul]^{l_I}\ar[dl]_{l_J}& C' \ar[ull]_{l_I'}\ar[dll]^{l_J'}\\
				      DJ
			      }}
	      \]
	      is commutative;
	\item a limit for $D$ is a terminal object in the category of cones for $D$. This means that given a cone for $D$, there is a unique arrow $k$ which is a morphism of cones.
\end{itemize}
Of course, a straightforward dualisation of diagram \eqref{itsalim} yields the definition of a cocone, and a colimit for $D$.

\subsection{Examples of limits and colimits}

\subsubsection{in monoids and Mon}

\subsubsection{in posets and Pos}

\subsubsection{in Set, and other algebraic categories}
\subsubsection{Exercises}
Exercises denoted with a star symbol are supposed to be difficult. Don't be put off, and enjoy!

\footnotesize
\begin{enumerate}
	\item 
\end{enumerate}
\end{document}